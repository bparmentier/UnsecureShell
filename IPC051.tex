\documentclass[french,10pt,a4paper]{article}
%\usepackage[latin1]{inputenc}
\usepackage[utf8]{inputenc}
\usepackage[francais]{babel}
% Modification des marges ------------------------------
\oddsidemargin -4mm % Decreases the left margin by 4mm
\textwidth 17cm %Sets text width across page = 17cm
\textheight 22cm %Sets text height up and down = 22cm
% -----------------------------------------------------
\usepackage[T1]{fontenc}
\usepackage{graphicx,color, caption2}
\usepackage{epsfig}
\usepackage{fancyhdr}
\pagestyle{fancy}
\usepackage{listings}
\usepackage{color}
\usepackage{makeidx}
\usepackage{textcomp} % pour le copyleft

% Valeurs par défaut le lstset
\lstset{language={},%C,Assembleur, TeX, tcl, basic, cobol, fortran, logo, make, pascal, perl, prolog, {}
    literate={â}{{\^a}}1 {ê}{{\^e}}1 {î}{{\^i}}1 {ô}{{\^o}}1 {û}{{\^u}}1
         {ä}{{\"a}}1 {ë}{{\"e}}1 {ï}{{\"i}}1 {ö}{{\"o}}1 {ü}{{\"u}}1
         {à}{{\`a}}1 {é}{{\'e}}1 {è}{{\`e}}1 {ù}{{\`u}}1 
         {Â}{{\^A}}1 {Ê}{{\^E}}1 {Î}{{\^I}}1 {Ô}{{\^O}}1 {Û}{{\^U}}1
         {Ä}{{\"A}}1 {Ë}{{\"E}}1 {Ï}{{\"I}}1 {Ö}{{\"O}}1 {Ü}{{\"U}}1
         {À}{{\`A}}1 {É}{{\'E}}1 {È}{{\`E}}1 {Ù}{{\`U}}1,
    commentstyle=\scriptsize\ttfamily\slshape, % style des commentaires
    basicstyle=\scriptsize\ttfamily, % style par défaut
    keywordstyle=\scriptsize\rmfamily\bfseries,% style des mots-clés
    backgroundcolor=\color[rgb]{.95,.95,.95}, % couleur de fond : gris clair
    framerule=0.5pt,% Taille des bords
    frame=trbl,% Style du cadre
    frameround=tttt, % Bords arrondis 
    tabsize=3, % Taille des tabulations
%   extendedchars=\true, % Incompatible avec utf8 et literate
    inputencoding=utf8,
    showspaces=false, % Ne montre pas les espaces 
    showstringspaces=false, % Ne montre pas les espaces entre ''
    xrightmargin=-1cm, % Retrait gauche 
    xleftmargin=-1cm, % Retrait droit
    escapechar=°}  % Caractère d'échappement, permet des commandes latex dans la source
% -----------------------------------------------------
%\makeindex
\begin{document}
\lhead{Labo IPC S.E 2$e$}
\rhead{Page \thepage}
\lfoot{\textcopyleft Bruno Parmentier (G38496) }
\rfoot{\today}
\cfoot{ }
\renewcommand{\footrulewidth}{0.4pt}

\setlength{\parindent}{0pt} % pas d'indentation

\lstset{frame=trBL}

\setcounter{tocdepth}{1}    % limiter les nivaux de table des matières
\setcounter{secnumdepth}{5} % La numérotation des sections au maximum

\newcommand{\titre}{Titre du sujet} % la variable contenant le titre du sujet


\title{\emph{Laboratoire\\\textbf{IPC}}}
\author{Bruno Parmentier (G38496)}
\date{31 mars 2014}
\maketitle
\thispagestyle{empty}
%\tableofcontents
% ~\\[5cm]
% \flushright{Dis, papa, ça veut dire quoi "Formating drive c" ?}
% \flushleft{ }
\newpage
%
\lstset{language={c}}
\renewcommand{\titre}{\textcolor{blue}{ IPC051 - pilotage à distance }}

\lhead{ \titre }
\section{{\titre} }

\begin{tabular}{|l|l|}
\hline
Titre : 	& \titre \\\hline
Support : 	& MDV2007 Installation Classique \\\hline
Date :		& 03/2014 \\\hline
\end{tabular}

\subsection{Énoncé}

On vous donne un shell simple. Modifiez-le afin d’y ajouter le pilotage à
distance. Les commandes entrées au clavier d’un processus client seront
communiquées au shell serveur. Les résultats des commandes (stdout et stderr)
seront affichées chez le client. Le shell ne traite qu’un client à la fois.

\subsection{Une solution}

\subsubsection{Serveur}
%\lstinputlisting{sources/ShellServeur.c}
Voir \verb#source/ShellServeur.c#.

\subsubsection{Client}
%\lstinputlisting{sources/ShellClient.c}
Voir \verb#sources/ShellClient.c#.

\subsection{Commentaires}

\begin{itemize}
    \item Le serveur ShellServeur est lancé. Il crée un socket et attend que
        le client se connecte.
    \item Le client ShellClient est lancé. Il crée un socket et demande une
        connexion au serveur.
    \item Une fois la connexion établie, le client envoie la commande lue au
        clavier. Celle-ci est reçue par le serveur qui l'exécute.
    \item Nous avons redirigé la sortie standard du serveur vers le descripteur
        de fichier donné par \verb#accept# (avec la fonction \verb#dup2#).
    \item Le résultat de la commande exécutée par le serveur est donc renvoyée
        au client via le socket.
\end{itemize}

\subsection{Bugs}
\begin{itemize}
    \item La taille du buffer de lecture du socket dans le client est définie à
        l'avance donc la fin du résultat de la commande peut être perdu.
    \item Lorsque le commande exécutée ne renvoie rien, le client tourne en
        boucle et il faut forcer l'arrêt de celui-ci avec \verb#Ctrl + C# puis
        faire un \verb#killall ShellServeur# pour arrêter le serveur.
    \item Quelques commandes testées~: \verb#ls -l#, \verb#cat Demo#, \verb#echo hello#, \verb#blrzgb#.
    \item Pas de gestion des pipes, redirections, etc. 
\end{itemize}
%\newpage
\end{document}
